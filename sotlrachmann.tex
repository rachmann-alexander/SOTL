\documentclass{article}
\usepackage[backend=biber]{biblatex}
\usepackage{biblatex-multiple-dm}
\addbibresource{references.bib}

\title{SoTL: Informatikübungen am Fachbereich Wirtschaftsingenieurwesen}
\author{Alexander Rachmann}
\date{\today}

\begin{document}

\maketitle

\section{Einleitung}
Der Bachelorstudiengang Wirtschaftsingenieurwesen wurde in 2024 so verändert, dass Informatik-Kompetenzen gestärkt werden. Dazu wird das bestehende Modul "Betriebliche Informatik" im Pflichtbereich aktualisiert; ein neuer Schwerpunkt wurde eingeführt und zwei betriebswirtschaftliche Wahlpflichtfächer werden eingeführt. Diese Veranstaltungen weisen einen unterschiedlich hohen Übungsanteil aus; aber in allen Veranstaltungen sind Übungen enthalten und prüfungsrelevant.

In allen Übungen tritt das Phänomen auf, dass die Teilnehmerzahl der Übungen über das Semester stark abnimmt. Die Veranstaltung "Betriebliche Informatik" hat eine relativ hohe Durchfallerquote; die Studierenden, die in den Übungen regelmäßig anwesend waren, haben typischerweise gute Noten. Dieses Phänomen besteht seit mehreren Jahren und ist dozentenunabhängig.

Unsere Hypothese ist, dass die Anwesenheit in den Übungen einen positiven Einfluss auf die Leistung in den Prüfungen hat. 

Mit Hilfe der SoTL-Förderung soll diese Hypothese genauer untersucht werden. Die Zielsetzung des Projekts ist eine höhere Teilnahme und dadurch eine bessere Leistung der Studierenden in den Prüfungen.



\section{Literaturübersicht}


Die untersuchte Literatur weist zwei zentrale Erkenntnisse auf
\cite{10.1145/2843043.2843061}\cite{10.1145/1539024.1508923}\cite{narula2013relationship}\cite{yao2011correlation}\cite{Hafeez_Khan_Jawaid_Haroon_2014}:

\begin{itemize}
    \item Die regelmäßige Teilnahme der Studierenden an Lehrveranstaltungen hat einen signifikanten positiven Einfluss auf ihre Leistung in Programmierkursen. Studierende, die häufiger am Unterricht teilnehmen, erzielen tendenziell bessere Noten, insbesondere in Informatikstudiengängen. Dies zeigt sich sowohl in Einführungsveranstaltungen als auch in fortgeschrittenen Programmierkursen. Zudem korrelieren frühe Abwesenheiten, wie das Versäumen der ersten Vorlesung, mit geringeren Durchschnittsnoten.
    \item Neben der direkten Wissensvermittlung spielt die Interaktion zwischen Studierenden eine zentrale Rolle für den Studienerfolg. Der Austausch mit Kommilitonen erweist sich als einer der stärksten Faktoren für die Motivation, ein Informatikstudium fortzusetzen. Daher sollten Informatikfakultäten klare Strukturen für studentische Interaktionen schaffen und Peer-Learning gezielt in den Lehrplan integrieren. Die Förderung der Anwesenheit kann somit nicht nur die akademische Leistung verbessern, sondern auch die langfristige Bindung an das Fach stärken.
    \item Die beiden häufigsten Gründe für eine niedrige Vorlesungsanwesenheit sind ungünstige Vorlesungszeiten und die Präferenz der Studierenden für Selbststudium oder Gruppenarbeit. Dennoch empfinden die meisten Studierenden das Lernen ohne Vorlesungsbesuch als schwierig und komplex.
\end{itemize}



\section{Untersuchungsplan}

Primnär untersuchte Formate:
\begin{itemize}
    \item 16 Betriebliche Informatik, IT-Praktikum (Sommersemester)
    \item 17f Software Engineering, Übung (Wintersemester)
\end{itemize}

Formate, in denen die Erkenntnisse direkt genutzt werden könnenn:
\begin{itemize}
    \item App Development
    \item ...
\end{itemize}






\printbibliography

\end{document}